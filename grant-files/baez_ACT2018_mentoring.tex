\documentclass[12pt,letterpaper]{amsart}

\sloppy

%\addtolength{\topmargin}{-70pt}
\addtolength{\textwidth}{60pt}
%\addtolength{\textheight}{120pt}
%\addtolength{\oddsidemargin}{-32pt}
\evensidemargin\oddsidemargin
\pagestyle{empty}
\pagenumbering{gobble}

\setlength{\parskip}{1ex plus 0.5ex minus 0.2ex}


%PACKAGES
\usepackage{geometry}               
\geometry{letterpaper}                   
\usepackage{graphicx}
\usepackage{amssymb}
%\usepackage{epstopdf}
\usepackage[all]{xy}
\usepackage{tikz}
\usetikzlibrary{arrows}
\usepackage[mathscr]{eucal}

\usepackage{rotating}
%\usepackage{enumerate}
%\usepackage{dsfont}
%\usepackage{mathtools}
%\usepackage{amsmath}
\usepackage{Trees}

%URL colors
\usepackage{color}
\definecolor{myurlcolor}{rgb}{0.6,0,0}
\definecolor{mycitecolor}{rgb}{0,0,0.8}
\definecolor{myrefcolor}{rgb}{0,0,0.8}
\usepackage{hyperref}
\hypersetup{colorlinks,
linkcolor=myrefcolor,
citecolor=mycitecolor,
urlcolor=myurlcolor}

\newtheorem{thm}{Theorem}
\newtheorem{cor}[thm]{Corollary}
\newtheorem{fact}[thm]{Fact}
\newtheorem{lem}[thm]{Lemma}
\newtheorem{example}[thm]{Example}
\newtheorem{defn}[thm]{Definition}
\newtheorem{prop}[thm]{Proposition}
\newtheorem{sch}[thm]{Scholium}
\newcommand{\ctr}[1]{\begin{center} #1 \end{center}}
\newcommand{\oo}{\infty}
\newcommand{\al}[1]{\begin{align} #1 \end{align}}
\newcommand{\alx}[1]{\begin{align*} #1 \end{align*}}
\renewcommand{\AA}{\mathbf{A}}
\newcommand{\RR}{\mathbb{R}}
\newcommand{\QQ}{\mathbb{Q}}
\newcommand{\NN}{\mathbb{N}}
\newcommand{\CC}{\mathbb{C}}
\newcommand{\ZZ}{\mathbb{Z}}
\newcommand{\CM}{\mathrm{CM}}
\newcommand{\gen}{\mathrm{gen}}
\newcommand{\maps}{\colon}
\newcommand{\tensor}{\otimes}
\renewcommand{\thefootnote}{\fnsymbol{footnote}}
\newcommand {\tuple}[1]{\langle #1 \rangle}
\newcommand{\define}[1]{{\bf \boldmath{#1}}}

\begin{document}

\ctr{\large MENTORING PLAN}
\title{Applied Category Theory 2018}
\author{John C.\ Baez}

\maketitle

One US postdoctoral researcher and eight US graduate students will be supported by this proposal. 
The postdoctoral researcher, Brendan Fong, is being intensively mentored by the one member 
of senior personnel in this proposal, David I.\ Spivak.  Both are at the Mathematics Department of
MIT, and they are doing research together.   

The graduate students supported by this proposal will receive extensive mentoring:

\begin{enumerate}
\item 
From January to April 2018 they will join an online reading seminar with 8 other graduate
students and 2 facilitators, one being Brendan Fong.  
The seminar will consist of 8 two-week blocks, 
each with one paper on applications of category theory as assigned reading, and with 
two students as leaders. Each block will have three phases: 

\begin{itemize}
\item a discussion of the assigned paper using an online videoconferencing system, consisting of a presentation by the leaders followed by questions,
\item
responses on a private forum from the other participants, moderated by the facilitators,
\item 
publication of an article summarizing the assigned paper on the well-known mathematics blog
\textsl{The $n$-Category Caf\'e}.
\end{itemize}

\vskip 1em
\item From April 23rd to 26th they will attend a research school at the Lorentz Center attended by 16 students, 2 facilitators, 4 mentors and a guest lecturer.   After opening lectures by the mentors, this will be devoted to collaborative research.  Each mentor will work with the four students who led the online discussions on the two papers they assigned.  They will try to prove conjectures outlined by the mentors at the beginning of the seminar. The results will be written up for publication.   Lectures will be videotaped and placed on the \textsl{Oxford Quantum Video} website as well as the conference website.
\vskip 1em

\item From April 30th to May 4th they will attend a workshop with 60 participants, also at the Lorentz Center. 
This focus will be on structured discussion, though there will also be three hours of invited 
talks per day.   They will join in discussions not only
with 30 senior researchers in applied category theory, but also 10 members of industry.  This 
will help them network and increase their ability to get jobs either in academia or in industry.
\end{enumerate}





\end{document}
